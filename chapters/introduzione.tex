
\chapter{Introduzione}
\label{cap:introduzione}
\begin{itemize}
  \item programmazione
  \item dsiufhs
  \item gsoigr
  \item dsfogih
\end{itemize}
\noindent Esempio di utilizzo di un termine nel glossario \\
\gls{api}. \\

\noindent Esempio di citazione in linea \\
\cite{site:agile-manifesto}. \\

\noindent Esempio di citazione nel pie' di pagina \\
citazione\footcite{womak:lean-thinking} \\

\section{L'azienda}

\begin{figure}[h]
    \centering
    \includegraphics[scale = 1.1]{../images/logo_ergon.png}
    \caption{Logo Ergon Informatica S.r.l.}
    \label{logo_ergon}
  \end{figure}

Ergon Informatica viene fondata nel 1988 come società di Ingegneria Informatica per applicazioni gestionali dedicate.
La società, che all'inizio conta solo alcuni dipendenti, si sviluppa in maniera costante negli anni e oggi può vantare 
una posizione di tutto rispetto tra le aziende dello stesso settore.

I clienti iniziali hanno giocato un ruolo fondamentale nello sviluppo del software prodotto da Ergon; essi infatti 
appartenevano per la maggior parte all'ambito alimentare e l'esperienza di queste prime installazioni ha permesso di 
acquisire delle competenze interne altamente specializzate in questo settore.
\\\\
Attualmente fanno parte della stessa gestione tre società:

\begin{itemize}
\item Ergon Informatica S.r.l. che si occupa del software;
\item Ergon S.r.l. che si occupa dei servizi tecnologici;
\item Ergon Servizi S.r.l. che si occupa dei servizi amministrativi, logistici e di marketing delle altre due.
\end{itemize}
  
\subsection*{Il Software}
Alla fine degli anni '90, a causa dell'imminenza di due eventi unici come l'anno 2000 e l'euro, la società riscrisse 
il software fino ad allora sviluppato, varando un grosso investimento di risorse e capitali per la riprogettazione in 
forma grafica della sua soluzione applicativa Ergdis .
Negli ultimi anni, l'avanzata delle nuove tecnologie su Web, ha stimolato l'azienda ad approfondire le sue conoscenze 
in questo ambito. Il risultato di questo sforzo è stato, da un lato la nascita di nuovi moduli che sono stati integrati 
in Ergdis, dall'altro la creazione di un settore interamente dedicato alla produzione di siti aziendali. 

Il nuovo software, studiato appositamente per la PMI, presenta le migliori caratteristiche di un moderno ERP.

Da qualche anno Ergon ha sviluppato nuovi moduli per la gestione CRM (Customer Relationship Management) che, 
utilizzando tecnologie Web, consentono di migliorare il rapporto con i clienti e conquistare nuovi segmenti di mercato. 
In collaborazione con l’ Università di Padova, è stata sviluppata inoltre una nuova procedura: “Prodector”, un software 
per la predizione degli ordini e la gestione delle scorte, che si avvale di moderne tecniche di Machine Learning.
Recentemente sono stati realizzati alcuni moduli per la visualizzazione in 3D del Magazzino (Ergvis), 
nonchè per la gestione geografica delle attività di Marketing (Geomarketing).

\subsection*{Servizi Hardware ed Assistenza}
La società completa la propria offerta con la vendita di prodotti hardware di primari marchi e la commercializzazione 
di programmi di terze parti, quali ad esempio la Tentata Vendita o la gestione dei fax,  integrati nella soluzione 
Ergdis, così da offrire un'unica interfaccia utente.
Ergon Informatica si occupa inoltre della sicurezza in ambito web,fornisce ed installa prodotti antivirus e firewall 
di primarie aziende presenti nel mercato, si occupa della virtualizzazione dei server e dei client presenti in azienda, 
installando e facendo assistenza sui prodotti di VmWare. 
Le sue soluzioni sono supportate da forti competenze tecnologiche su: xSeries, Linux, ambiente MS Windows, Networking, 
Storage, Wireless, Web Services.
Attualmente Ergon fornisce una rete di oltre 200 clienti distribuita in tutta Italia.
\section{L'idea}

Introduzione all'idea dello stage.

\section{Organizzazione del testo}

\begin{description}
    \item[{\hyperref[cap:processi-metodologie]{Il secondo capitolo}}] descrive ...
    
    \item[{\hyperref[cap:descrizione-stage]{Il terzo capitolo}}] approfondisce ...
    
    \item[{\hyperref[cap:analisi-requisiti]{Il quarto capitolo}}] approfondisce ...
    
    \item[{\hyperref[cap:progettazione-codifica]{Il quinto capitolo}}] approfondisce ...
    
    \item[{\hyperref[cap:verifica-validazione]{Il sesto capitolo}}] approfondisce ...
    
    \item[{\hyperref[cap:conclusioni]{Nel settimo capitolo}}] descrive ...
\end{description}

Riguardo la stesura del testo, relativamente al documento sono state adottate le seguenti convenzioni tipografiche:
\begin{itemize}
	\item gli acronimi, le abbreviazioni e i termini ambigui o di uso non comune menzionati vengono definiti nel glossario, situato alla fine del presente documento;
	\item per la prima occorrenza dei termini riportati nel glossario viene utilizzata la seguente nomenclatura: \emph{parola}\glsfirstoccur;
	\item i termini in lingua straniera o facenti parti del gergo tecnico sono evidenziati con il carattere \emph{corsivo}.
\end{itemize}
