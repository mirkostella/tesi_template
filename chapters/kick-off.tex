\chapter{Descrizione dello stage}
\label{cap:descrizione-stage}

In questo capitolo verrà descritto lo stage, gli obiettivi attesi, l'analisi dei requisiti e la pianificazione del lavoro antecedente all'inizio dello stage in 
modo da fornire i concetti fondamentali per comprendere al meglio i capitoli che seguiranno.
\section{Introduzione al progetto}
%intro
La gestione del magazzino per lo stoccaggio delle merci assume un ruolo chiave quando si vogliono aumentare 
le performance aziendali sia in fase di entrata merce sia nella fase di preparazione degli ordini 
cliente da evadere. Un corretto stoccaggio delle merci nelle ubicazioni di magazzino consente infatti di ridurre i tempi di 
preparazione degli ordini e di rendere più efficienti i propri magazzinieri.

%scopo del progetto
Il progetto si propone di sviluppare un sistema che andrà ad integrarsi al WMS e dovrà supportarlo nella definizione delle ubicazioni 
più opportune per lo stoccaggio dei vari articoli in modo da ottimizzare i tempi di preparazione degli ordini.
% considerazioni
Questo dovrà essere fatto tenendo conto sia dell’efficienza che si vuole ottenere nel prelievo degli articoli sia rispettando regole e condizioni imposte dal contesto.
Infatti, la disposizione degli articoli all'interno del magazzino è fortemente influenzata dai vincoli fisici dovuti agli articoli, dal mezzo con cui questi vengono prelevati, dalla
stagionalità di vendita e dal fatto che le correlazioni tra articoli potrebbero variare nel tempo.
A causa della variazione dei vincoli potrebbe essere necessaria una nuova disposizione degli articoli a magazzino che verrà effettuata su 
decisione del cliente seguendo le indicazioni del sistema solo quando risulta conveniente in quanto la riorganizzazione del magazzino è un processo dispendioso e anche 
portando ad una migliore disposizione degli articoli i costi di riorganizzazione potrebbero essere maggiori dei ricavi che si otterrebbero dall'incremento dell'efficienza di prelievo.

\section{Analisi preventiva dei rischi}
Durante la fase di analisi iniziale sono stati individuati alcuni possibili rischi a cui si potrà andare incontro.
Si è quindi proceduto a elaborare delle possibili soluzioni per far fronte a tali rischi.\\
\begin{risk}{Inesperienza tecnologica}
    \riskdescription{Le tecnologie da utilizzare sono sconosciute ed è richiesto del tempo per il loro apprendimento}
    \risksolution{
        \begin{itemize}
            \item lettura della documentazione ufficiale delle tecnologie coinvolte e assistenza da parte del tutor per velocizzare il processo di apprendimento;
            \item svolgere dei tutorial su problemi semplici prima di utilizzare la tecnologia di interesse sul problema riguardante il progetto;
            \item individuare quali sono le parti che caratterizzano maggiormente la tecnologia e quali sono le operazioni più ricorrenti gestite da tale tecnologia.
        \end{itemize}
    }
    \label{risk:inesperienza-tecnologica} 
\end{risk}
\begin{risk}{Contesto applicativo sconosciuto}
    \riskdescription{Il contesto della gestione di magazzino e tutte le operazioni coinvolte sono sconosciute}
    \begin{itemize}
        \item studio individuale dei concetti principali sulla gestione di magazzino da documenti forniti dal tutor e studiando da fonti attendibili presenti nel web;
        \item prevedere una trasferta dal cliente per discutere sulla morfologia del magazzino, mezzi utilizzati, operazioni effettuate e altri vincoli;
        \item comunicazione tra tutor e cliente e successivo riferimento delle nuove informazioni utili al tirocinante;
        \item tenere aggiornato il glossario di progetto per comprendere al meglio la terminologia utilizzata dai più esperti del settore ed evitare fraintendimenti.
    \end{itemize}
    \label{risk:contesto-applicativo-sconosciuto} 
\end{risk}
\begin{risk}{Incomprensioni dei requisiti}
    \riskdescription{I requisiti potrebbero non venire compresi a pieno portando a divergere dalle aspettative del cliente o dagli obiettivi di progetto}
    \begin{itemize}
        \item confronti frequenti tra tutor e tirocinante in modo da limitare le incomprensioni;
        \item alla fine di ogni discussione tra tutor e tirocinante fare un breve resoconto verbale dei concetti principali discussi.
    \end{itemize}
    \label{risk:incomprensioni-dei-requisiti} 
\end{risk}
\begin{risk}{Assenza del tutor in azienda}
    \riskdescription{Il tutor potrebbe essere assente}
    \risksolution{contattare il tutor tramite il recapito telefonico o mail forniti in precedenza se necessario.}
    \label{risk:assenza-del-tutor-in-azienda} 
\end{risk}
\begin{risk}{Ritardi sulla tabella di marcia pianificata}
    \riskdescription{Alcune delle attività pianificate potrebbero richiedere più tempo del previsto}
    \begin{itemize}
        \item svolgere per prime le attività più importanti che permettono di soddisfare i requisiti obbligatori;
        \item scegliere cosa escludere da quanto pianificato nel caso ci si rendesse conto che la mole di lavoro supera le risorse a disposizione;
        \item contattare il cliente per ridefinire le sue priorità se necessario.
    \end{itemize}
    \label{risk:ritardi-sulla-tabella-di-marcia-pianificata} 
\end{risk}
\begin{risk}{Cambiamento dei requisiti in corso d'opera}
    \riskdescription{Alcuni requisiti stabiliti potrebbero subire delle variazioni}
    \begin{itemize}
        \item utilizzo di un sistema di versionamento per la gestione dello storico del progetto in modo da riportarlo ad una versione precedente se si dovesse ritenere utile;
        \item scegliere cosa escludere da quanto pianificato nel caso ci si rendesse conto che la mole di lavoro supera le risorse a disposizione;
        \item contattare il cliente per ridefinire le sue priorità;
        \item utilizzare le buone pratiche per lo sviluppo di un progetto imparate durante il corso di studi per facilitare le modifiche non previste. In particolare durante il periodo di 
        progettazione utilizzare le metodologie opportune per minimizzare le dipendenze tra le parti che compongono il sistema.
    \end{itemize}
    \label{risk:cambiamento-dei-requisiti-in-corso-opera} 
\end{risk}
\begin{risk}{Basse performance del sistema in sviluppo}
    \riskdescription{Il calcolo della disposizione degli articoli a magazzino da parte del sistema in svilupppo potrebbe richiedere molto tempo a causa dell'elevato numero articoli coinvolti}
    \begin{itemize}
        \item salvare le informazioni ricorrenti a database in modo che il sistema non debba ricalcolarle ad ogni esecuzione;
        \item migliorare le query che richiedono grandi quantità di dati al database valutando se è possibile semplificarle per evitare operazioni onerose. Valutare l'utilizzo degli indici.
    \end{itemize}
    \label{risk:basse-performance-del-sistema-in-sviluppo} 
\end{risk}
\begin{risk}{Presenza di malfunzionamenti nel sistema}
    \riskdescription{La rilevazione degli errori a runtime potrebbe essere difficoltosa in alcuni casi}
    \begin{itemize}
        \item scomporre il problema in sottoproblemi in modo da isolare il più possibile la causa del malfunzionamento;
        \item creare delle eccezioni specifiche che mostrino a schermo dei messaggi di errore in modo da individuare più facilmente la causa del malfunzionamento;
        \item utilizzare il debugger per individuare anomalie che possono far intuire la causa del malfunzionamento. Utilizzare i breakpoint in modo da interrompere l'esecuzione del 
        programma nel punto in cui è presente il malfuzionamento per capirne la causa.
    \end{itemize}
    \label{risk:presenza-di-malfunzionamenti-nel-sistema}
\end{risk}

\section{Requisiti e obiettivi}
Durante la stesura del piano di lavoro necessaria all'avvio dello stage sono stati fissati i requisiti di progetto.
I requisiti di seguito riportati hanno un codice requisito che ne permette l'identificazione, una descrizione e sono classificati in base alla loro importanza.\\
La classificazione comprende le seguenti diciture:\\\\
\textbf{OB}: rappresentano i requisiti obbligatori, vincolanti, che dovranno necessariamente essere soddisfatti;\\
\textbf{DE}: rappresentano i requisiti desiderabili, non vincolanti, ma dal riconoscibile valore aggiunto;\\
\textbf{FA}: rappresentano i requisiti facoltativi, rappresentanti valore aggiunto non strettamente competitivo.\\

\begin{table}
    \centering
    \begin{tabularx}{\textwidth}{|Y|Z|Y|}
    \hline
    \textbf{Requisito} & \textbf{Descrizione} & \textbf{Classificazione}\\
    \hline
    R1 & Sviluppo programmi per inserimento dei vincoli & OB \\
    \hline
    R2 & Sviluppo programma per la lettura e preparazione dei dati dai movimenti di magazzino & OB \\
    \hline
    R3 & Sviluppo del sistema di individuazione delle correlazioni tra i vari articoli e proposta delle ubicazioni & OB \\
    \hline
    R4 & Acquisizione di competenze sull’utilizzo di algoritmi di Ricerca Operativa e Machine Learning e applicazione in un caso reale & OB \\
    \hline
    R5 & Visualizzazione 2D del magazzino e delle ubicazioni suggerite & DE \\
    \hline
    R6 & Visualizzazione attraverso dashboard e oggetti grafici dei dati raccolti ed elaborati utilizzati per definire la proposta suggerita & DE \\
    \hline
    R7 & Utilizzo del multithreading nelle fasi in cui è richiesta una maggiore capacità di calcolo & FA \\
    \hline
    \end{tabularx}
    \caption{Tabella del tracciamento dei requisiti di progetto}
    \end{table}

\section{Pianificazione}

\begin{table}
    \centering
    \begin{tabularx}{\textwidth}{|Y|Z|}
    \hline
    \textbf{Durata in ore} & \textbf{Descrizione attività} \\
    \hline
    24 & Analisi dello user case, del sistema ERP e definizione del problema \\
    \hline
    20 & Analisi dei requisiti e stesura della relativa documentazione \\
    \hline
    32 & Studio delle tecnologie aziendali necessarie allo sviluppo del 
    modulo (linguaggio di programmazione C\#, .NET Framework, 
    componenti DevExpress, database Informix e altre tecnologie) \\
    \hline
    80 & Studio di algoritmi e tecniche applicabili nel caso di studio (es.
    ambito Ricerca operativa e Machine Learning) \\
    % \hline
    % 112 & Sviluppo delle componenti:
    % \begin{itemize}
    %     \item Sviluppo programmi per l’inserimento dei vincoli;
    %     \item Sviluppo procedura di reperimento e preparazione dei 
    %     dati di magazzino;
    %     \item Sviluppo dell’algoritmo per l’individuazione delle 
    %     correlazioni tra le varie referenze e proposta delle 
    %     ubicazioni. \\
    % \end{itemize}
    \hline
    20 & Validazione progetto \\
    \hline
    24 & Stesura della documentazione del prodotto sviluppato \\
    \hline
    \end{tabularx}
    \caption{Tabella del tracciamento dei requisiti di progetto}
    \end{table}