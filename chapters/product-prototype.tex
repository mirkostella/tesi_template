\chapter{Progettazione e codifica}
\label{cap:progettazione-codifica}

\intro{Breve introduzione al capitolo}\\

\section{Tecnologie e strumenti}
\label{sec:tecnologie-strumenti}

Di seguito viene data una panoramica delle tecnologie e strumenti utilizzati.

\subsection*{Tecnologia 1}
Descrizione Tecnologia 1.

\subsection*{Tecnologia 2}
Descrizione Tecnologia 2

\section{Ciclo di vita del software}
\label{sec:ciclo-vita-software}

\section{Progettazione}
\label{sec:progettazione}

\subsection{Classi modello di business} %**************************
Il modello di business costituisce la logica dell'applicazione.
Le classi che ne fanno parte sono:

\subsection{Classi utili al modello di business} %**************************
Di seguito sono riportate le classi che vengono utilizzate dal modello di business.
\\\\
\textbf{Magazzino}
\\
Rappresenta un magazzino sul quale può essere eseguita una \gls{simulazione di prelievog} o i
test sul numero di fermate e sulla distanza percorsa per completarla.
Un magazzino contiene gli articoli che vengono disposti nei punti di prelievo a seconda del passo associazione o della disposizione degli articoli presenti all'interno del magazzino cliente.
Un punto di prelievo può contenere un solo articolo.
Tramite il carrello vengono prelevati gli articoli fino al completamento della missione di prelievo.
La posizione del carrello all'interno del magazzino viene indicata dalla posizione in cui si trova la sua ultima sezione che è la sezione ad indice minore rispetto alle posizioni di prelievo.\\\\
\textbf{Dipendenze}:
\begin{itemize}
    \item Carrello (Associazione): utilizzata per svolgere le simulazioni di prelievo;
    \item Articolo (Associazione): utilizzata per inizializzare i punti di prelievo.\\
\end{itemize}
\textbf{I metodi principali della classe sono}:
\\\\
\textit{Costruttori}:
\begin{itemize}
    \item Magazzino(in dim:int, in car:Carrello)\\
    Crea un'istanza di un magazzino con dim punti di prelievo e utilizza il carrello car per scorrere e prelevare gli articoli dai punti di prelievo;\\
    dim: dimensione del magazzino che corrisponde al numero di ubicazioni e quindi di articoli che possono essere stoccati all'interno dei punti di prelievo;\\
    car: carrello utilizzato per scorrere i punti di prelievo e prelevarne gli articoli.\\
\end{itemize}
\textit{Metodi per popolare il magazzino}:
\begin{itemize}
    \item PopolaMagazzino(in articoli:List<Articolo>): void\\
    Inizializza i punti di prelievo con gli elementi del parametro articoli seguendo l'ordine in cui si trovano all'interno della lista;
    \item PopolaMagazzino(in passo:int, in articoli:List<Articolo>, in strut\_associati: Dictionary<string-Dictionary<string-int> >): void\\
    Inizializza i punti di prelievo con gli elementi della lista articoli seguendo il passo fornito e la strut\_associati.\\
    passo: rappresenta il numero di articoli associati da disporre nelle posizioni contigue dei punti di prelievo.\\
    articoli: rappresenta gli articoli da utilizzare per inizializzare i punti di prelievo.\\
    strut\_associati: dizionario che ha come chiavi i codici univoci degli articoli presenti nella lista articoli e come valori i dizionari aventi come chiavi 
    i codici articoli che sono associati all'articolo chiave e come valori il numero di volte che l'associato compare all'interno di un ordine cliente assieme all'articolo chiave.\\
\end{itemize}
\textit{Metodi per la gestione del carrello}:
\begin{itemize}
    \item GetPosCarrello(): int\\
    Ritorna la posizione della sezione ad indice minore del carrello;
    \item GetPuntiPrelievoCarrello(): Array<Articolo>\\
    Ritorna gli articoli che corrispondono al range di prelievo del carrello quindi gli articoli che sono prelevabili dalla posizione in cui si trova il carrello;
    \item CaricaCoperturaCarrello(): void\\
    Carica nelle corrispettive sezioni del carrello gli articoli che sono target nella missione in corso e che sono all'interno della copertura del carrello;
    \item GetCoperturaCarrello(): List<int>\\
    Ritorna la lista di indici delle posizioni di prelievo all'interno della copertura del carrello;
    \item MuoviCodaCarrello(in pos:int): void\\
    Sposta la coda del carrello alla posizione indicata dal parametro pos.\\
\end{itemize}
\textit{Metodi per l'esecuzione dei test}:
\begin{itemize}
    \item TestValutazioneFermate(): int\\
    Ritorna il numero di fermate impiegate dal carrello per completare la missione di prelievo che gli è stata assegnata;
    \item TestValutazioneDistanza(): int\\
    Ritorna la distanza percorsa dal carrello per completare la missione di prelievo che gli è stata assegnata.\\
\end{itemize}
\textit{Metodi privati di utilità}:
\begin{itemize}
    \item ControlloArticoliMissionePresenti(): bool\\
    Ritorna true se tutti gli articoli coinvolti nella missione assegnata al carrello sono presenti all'interno dei punti di prelievo altrimenti false.
    Nel caso ritorni true viene garantito che la missione assegnata al carrello possa essere completata con successo;
    \item GetIndiceProssimoArtInMissioni(): int\\
    Ritorna l'indice del prossimo articolo da prelevare per la missione assegnata al carrello;
    \item GetMigliorAssociato(in cod:string, in strut:Dictionary<string-int>): string\\
    Metodo di utilità utilizzato principalmente durante il popolamento del magazzino.\\
\end{itemize}       
\textbf{Carrello}\\
Rappresenta il carrello fisico che si muove all'interno di un magazzino per prelevare gli articoli dai punti di prelievo.
Il numero di posti corrisponde al numero di sezioni e ogni sezione può contenere gli articoli riguardanti un solo ordine cliente.
Dato che l'obiettivo del progetto è focalizzato sullo stoccaggio della merce è stato assunto che tutti gli articoli appartenenti ad un ordine 
da completare possano essere collocati nelle sezioni relative senza tenere conto di vincoli fisici.
La classe presenta due contatori che tengono conto della distanza percorsa (espressa in punti di prelievo visitati) e il numero di fermate effettuate
(il punto da cui parte il carrello non viene contata).\\
\textbf{Dipendenze}:
\begin{itemize}
    \item Missione (Associazione): corrisponde alla missione assegnata al carrello durante la simulazione di prelievo.\\
\end{itemize}
\textbf{I metodi principali della classe sono}:\\
\textit{Costruttori}:\\
\begin{itemize}
    \item Carrello(in n\_posti:int, in mis:Missione):\\
    Crea un'istanza di un carrello con n\_posti e gli assegna la missione mis.\\
\end{itemize}
\textit{Metodi pubblici}:\\
\begin{itemize}
    \item Carica(in cod\_art:string, in pos\_prelievo:int): void\\
    Imposta la posizione di prelievo del target missione corrispondente al cod\_art alla pos\_prelievo;
    \item Pieno(): bool\\
    Ritorna true se tutte le sezioni del carrello sono occupate da almeno un articolo ovvero quando il numero di posti occupati 
    corrisponde al numero di posti totale, altrimenti false;
    \item Svuota(): void\\
    Imposta il numero di posti occupati a zero;
    \item Reset(): void\\  
    Imposta il numero di posti occupati,il numero di fermate effettuate e la distanza percorsa a zero; inoltre ripristina la missione che gli è stata assegnata.\\
\end{itemize} 
\textbf{Missione}\\
Rappresenta la missione assegnata al carrello che deve essere completata durante la simulazione di prelievo prelevando i target missione dai punti di prelievo.\\
\textbf{Dipendenze}:
\begin{itemize}
    \item Target (Associazione): un target corrisponde ad un articolo da prelevare per completare la missione.
\end{itemize}
\textbf{I metodi principali della classe sono}:\\
\textit{Costruttori}:\\
\begin{itemize}
    \item Missione(in targets:List<Target>)\\
    Crea un'istanza di una missione con lista target uguale a targets parametro formale;
    \item Missione(in missione:Missione)\\
    Costruttore di copia profonda.\\
\end{itemize}
\textit{Metodi pubblici}:\\
\begin{itemize}
    \item SetPosPrelievoTarget(in cod\_art:string, in pos\_prelievo:int): void\\
    Imposta la posizione di prelievo a pos\_prelievo dell'articolo con codice articolo cod\_art nella lista dei target;
    \item MissioneCompletata(): bool\\
    Ritorna true se tutti i target missione all'interno della lista target hanno una posizione di prelievo diversa da -1 altrimenti false.
    Un target con posizione di prelievo diversa da -1 è stato prelevato dalla posizione di prelievo indicata dal valore impostato.
    \item Ripristina(): void\\
    Imposta tutte le posizioni di prelievo dei target missione a -1.
    \item GetTargets(): List<Target>\\
    Ritorna la lista dei target missione.\\
\end{itemize} 
\textbf{Target}\\
Rappresenta un articolo da prelevare per completare una missione di prelievo.
Un target viene identificato dal codice articolo da prelevare, dall'ordine a cui appartiene (numero ordine e anno ordine) e dalla posizione
di prelievo da cui è stato prelevato (se non ancora prelevato la posizione di prelievo ha valore -1)\\
\textbf{I metodi principali della classe sono}:\\
\textit{Costruttori}:\\
\begin{itemize}
    \item Target(in cod\_art:string, in nr\_ord:int, in anno\_ord:int)\\
    Crea un'istanza di un target assegnando ai campi dati corrispondenti i parametri formali;
    \item Target(in target:Target)\\
    Costruttore di copia profonda.\\
\end{itemize}
\textit{Metodi pubblici}:\\
I metodi che offre la classe sono i metodi getter e setter per ricavare o impostare i valori dei campi dati.\\
\textbf{ErgdisEntities}\\
Classe che contiene i riferimenti agli oggetti che rappresentano le entità di interesse fornite dal DB.
Viene utilizzata per interfacciarsi al DB.
Tutte le dipendenze di cui necessita la classe vengono generate automaticamente da Entity Framework. Per aggiungere un'entità basta aggiornare
la classe utilizzando le funzionalità offerte dall'IDE di sviluppo tramite interfaccia grafica.
\textbf{Dipendenze}:
\begin{itemize}
    \item DettagliOrdiniCliente (Associazione): \\
    Classe che rappresenta i dati presenti a database nella tabella DettagliOrdiniCliente.\\
    La tabella è strutturata in modo che ogni record contiene le informazioni per un articolo presente in un ordine cliente in modo da
    fornirne il codice, il numero dell'ordine e l'anno dell'ordine.
    \item AnagraficaArticoli (Associazione): \\
    Classe che rappresenta i dati presenti a database nella tabella AnagraficaArticoli.\\
    La tabella è strutturata in modo che ogni record contiene le informazioni utili a reperire le informazioni relative ad ogni singolo articolo (codice articolo,descrizione,unità di misura ecc.).
    \item AnagraficaArticoliMagazzinoCliente (Associazione): \\
    Classe che rappresenta i dati presenti a database nella tabella AnagraficaArticoliMagazzinoCliente.\\
    La tabella è strutturata in modo che ogni record contiene le informazioni utili a reperire le informazioni relative ad ogni singolo articolo in relazione con il magazzino cliente.
    Ad esempio un dato utile da ricavare da questa tabella potrebbe essere il codice del deposito in cui è stoccato l'articolo.
    \item FrequenzeArticoli (Associazione): classe che rappresenta i dati presenti a database nella tabella FrequenzaArticoli.
    Tiene traccia dei codici articoli e delle frequenze di vendita di ciascuno ovvero quante volte gli articoli compaiono all'interno degli 
    ordini cliente. La classe permette di velocizzare i tempi di caricamento a runtime dell'applicazione in quanto avendo salvato a database
    le frequenze di vendita non è necessario ricalcolarle ogni volta che si avvia l'applicazione. Questo ha portato ad un aumento evidente delle 
    performance durante il caricamento iniziale della finestra principale;\\
    \item AssociazioniArticoli (Associazione): classe che rappresenta i dati presenti a database nella tabella AssociazioniArticoli.
    Presenta i campi cod\_art e cod\_art\_assoc e con il campo freq\_assoc viene memorizzato il numero di volte che il primo articolo è
    presente all'interno degli ordini clienti assieme al secondo articolo.
    Anche questa classe permette di velocizzare i tempi di caricamento evitando di ricalcolare ad ogni esecuzione le frequenze di associazione
    tra gli articoli.\\
    \item Database (Associazione): classe utilizzata per interagire con il database.\\
\end{itemize}

\textbf{ComparatoreDiffPassiMagazzini}\\
Classe utilizzata per confrontare i risultati dei test eseguiti tra configurazioni diverse di magazzini.
\textbf{Dipendenze}:
\begin{itemize}
    \item Magazzino (Associazione): rappresenta i magazzini da confrontare.\\
\end{itemize}
\textbf{I metodi principali della classe sono}:\\
\textit{Costruttori}:\\
\begin{itemize}
    \item ComparatoreDiffPassiMagazzini(in magazzino1:Magazzino, in magazzino2:Magazzino, in inizio\_range:int, in fine\_range:int):\\
    Crea un'istanza di ComparatoreDiffPassiMagazzini con magazzino1 e magazzino2 da confrontare ed imposta il range di ubicazioni all'interno dei magazzini da confrontare secondo i parametri inizio\_range e fine\_range.\\ 
\end{itemize}
\textit{Metodi pubblici}:\\
\begin{itemize}
    \item GetDiffPosizioni(): int \\
    Ritorna il numero di ubicazioni che non contengono gli stessi articoli agli indici corrispondenti nei punti prelievo.
    Questo metodo viene utilizzato per confrontare configurazioni di magazzino che hanno passi associazione diversi in modo da poter capire quanti articoli non si trovano nelle stesse posizioni
    rispetto a passi associazioni diversi;\\
    \item GetDiffArticoli(): int \\
    Ritorna il numero di articoli che non sono presenti nel primo magazzino e sono presenti all'interno del secondo magazzino.
    Questo metodo viene utilizzato per confrontare configurazioni di magazzino che hanno passi diversi in modo da poter capire quanti articoli non si trovano all'interno del range in passi associazioni diversi;\\
    \item GetCodArtRange(): List<string> \\
    Ritorna i codici degli articoli che si trovano all'interno del range di confronto (tra inizio\_range e fine\_range compresi).\\
\end{itemize} 
\subsection{Classi utili alle viste}
Di seguito sono riportate le classi che vengono utilizzate dalle viste per rappresentare le informazioni di interesse per l'utente.
\\\\
\textbf{ResocontoTest}\\
Classe astratta che rappresenta un set di test eseguiti e ne raccoglie i risultati ottenuti ed altre informazioni tra le quali il nome e una descrizione testuale dei test eseguiti.
\textbf{Estesa da}:
\begin{itemize}
    \item ResocontoTestCliente: raccoglie i test eseguiti sul magazzino cliente senza passo associazioni;
    \item ResocontoTestErgon: raccoglie i risultati dei test sul magazzino Ergon con passo associazioni.\\
\end{itemize}
\textit{Costruttori}:\\
\begin{itemize}
    \item ResocontoTest(): utilizzato dalle sottoclassi per inizializzarne i campi dati.\\
\end{itemize}
\textit{Metodi pubblici}:\\
I metodi messi a disposizione sono i getter e setter per il nome e la descrizione dei test presenti all'interno del resoconto.\\
\textbf{ResocontoTestCliente}\\
Classe che contiene i test effettuati sul magazzino cliente e ne contiene le informazioni di interesse.\\
\textbf{Estende}:\\
ResocontoTest\\
\textbf{Dipendenze}:
\begin{itemize}
    \item RisultatoTestCliente (Associazione): risultato di un test effettuato sul magazzino cliente.\\
\end{itemize}
\textit{Costruttori}:\\
\begin{itemize}
    \item ResocontoTestCliente(in risultati:List<RisultatoTestCliente>):\\
    Crea un'istanza di ResocontoTestCliente che contiene i risultati dei test effettuati sul magazzino cliente.\\
\end{itemize}
\textit{Metodi pubblici}:\\
I metodi pubblici della classe sono i getter e i setter per ricavare i valori dei campi dati.\\
\textbf{ResocontoTestErgon}\\
Classe che contiene i test effettuati sul magazzino Ergon e ne contiene le informazioni di interesse.\\
\textbf{Dipendenze}:
\begin{itemize}
    \item RisultatoTestErgon (Associazione): risultato di un test effettuato sul magazzino Ergon quindi con passo associazioni articoli.\\
\end{itemize}
\textit{Costruttori}:\\
\begin{itemize}
    \item ResocontoTestErgon(in risultati:List<RisultatoTestErgon>):\\
    Crea un'istanza di ResocontoTestErgon che contiene i risultati dei test effettuati sul magazzino Ergon.
\end{itemize}
\textit{Metodi pubblici}:\\
\begin{itemize}
    \item I metodi pubblici della classe sono i getter e i setter per ricavare i valori dei campi dati.\\
\end{itemize} 
\textbf{RisultatoTestCliente}\\
Un RisultatoTestCliente rappresenta il risultato di un test avvenuto sul magazzino cliente senza passo associazioni utilizzando un carrello  per effettuare i prelievi 
dalle posizioni di magazzino.
Il risultato ha come campi dati il nome del cliente e il codice del deposito su cui avviene il test.\\
\textbf{Estende}:\\
RisultatoTestCarrello\\
\textbf{Implementa}:\\
RisultatoTestCarrello\\
\textit{Costruttori}:\\
\begin{itemize}
    \item RisultatoTestCliente(in nome\_cliente:string, in cod\_dep:string, in dimCarrello:int).\\
    Crea un'istanza di RisultatoTestCliente ed imposta i campi dati con i parametri passati come parametri formali.\\
\end{itemize}
\textit{Metodi pubblici}:\\
\begin{itemize}
    \item I metodi pubblici della classe sono i getter e i setter per ricavare i valori dei campi dati.\\
\end{itemize} 
\textbf{RisultatoTestDistanzaCliente}\\
Rappresenta un risultato di un test distanza effettuato sul magazzino cliente e ne memorizza la distanza percorsa dal carrello per completare il test.\\
\textbf{Estende}:\\
RisultatoTestCliente\\
\textbf{Implementa}:\\
RisultatoTestDistanza\\
\textit{Costruttori}:\\
\begin{itemize}
    \item RisultatoTestDistanzaCliente(in nome\_cliente:string, in cod\_dep:string, in dimCarrello:int).\\
    Crea un'istanza di RisultatoTestDistanzaCliente ed imposta i campi dati con i parametri passati come parametri formali.\\
\end{itemize}
\textit{Metodi pubblici}:\\
\begin{itemize}
    \item I metodi pubblici della classe sono i getter e i setter per ricavare i valori dei campi dati.\\
\end{itemize} 

\textbf{RisultatoTestFermateCliente}\\
Rappresenta un risultato di un test fermate effettuato sul magazzino cliente e ne memorizza il numero di fermate impiegate dal carrello per completare il test.\\
\textbf{Estende}:\\
RisultatoTestCliente\\
\textbf{Implementa}:\\
RisultatoTestFermate\\
\textit{Costruttori}:\\
\begin{itemize}
    \item RisultatoTestFermateCliente(in nome\_cliente:string, in cod\_dep:string, in dimCarrello:int).\\
    Crea un'istanza di RisultatoTestFermateCliente ed imposta i campi dati con i parametri passati come parametri formali.\\
\end{itemize}
\textit{Metodi pubblici}:\\
\begin{itemize}
    \item I metodi pubblici della classe sono i getter e i setter per ricavare i valori dei campi dati.\\
\end{itemize}


% \textbf{nome_classe}\\
% descrizione
% \textbf{Dipendenze}:
% \begin{itemize}
%     \item .\\
% \end{itemize}
% \textit{Costruttori}:\\
% \begin{itemize}
%     \item .\\
% \end{itemize}
% \textit{Metodi pubblici}:\\
% \begin{itemize}
%     \item .\\
% \end{itemize} 

% \item RisultatoTestCliente(in nome\_cliente:string, in cod_dep:string, in dimCarrello:int):\\
% Crea un'istanza di RisultatoTestCliente utilizzando il nome del cliente, il codice del deposito sul quale viene effettuato il test e la 
% dimensione del carrello utilizzato per completare gli ordini cliente per settare i relativi campi dati della classe.






\section{Design Pattern utilizzati}

\section{Codifica}
