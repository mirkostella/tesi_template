% Acronyms
\newacronym[description={\glslink{apig}{Application Program Interface}}]
    {api}{API}{Application Program Interface}

\newacronym[description={\glslink{umlg}{Unified Modeling Language}}]
{uml}{UML}{Unified Modeling Language}

\newacronym[description={\glslink{WMSg}{Warehouse Management System}}]
{wms}{WMS}{Warehouse Management System}

% Glossary entries
\newglossaryentry{apig} {
    name=\glslink{api}{API},
    text=Application Program Interface,
    sort=api,
    description={in informatica con il termine \emph{Application Programming Interface API} (ing. interfaccia di programmazione di un'applicazione) si indica ogni insieme di procedure disponibili al programmatore, di solito raggruppate a formare un set di strumenti specifici per l'espletamento di un determinato compito all'interno di un certo programma. La finalità è ottenere un'astrazione, di solito tra l'hardware e il programmatore o tra software a basso e quello ad alto livello semplificando così il lavoro di programmazione}
}

\newglossaryentry{umlg} {
    name=\glslink{uml}{UML},
    text=UML,
    sort=uml,
    description={in ingegneria del software \emph{UML, Unified Modeling Language} (ing. linguaggio di modellazione unificato) è un linguaggio di modellazione e specifica basato sul paradigma object-oriented. L'\emph{UML} svolge un'importantissima funzione di ``lingua franca'' nella comunità della progettazione e programmazione a oggetti. Gran parte della letteratura di settore usa tale linguaggio per descrivere soluzioni analitiche e progettuali in modo sintetico e comprensibile a un vasto pubblico}
}
\newglossaryentry{WMSg} {
    name=\glslink{wms}{WMS},
    text=WMS,
    sort=wms,
    description={ nel contesto della gestione di magazzino \emph{Warehouse Management System} (ing. sistema di gestione di magazzino) è una soluzione software che offre visibilità sull'intero inventario aziendale e gestisce le operazioni di evasione degli ordini dal centro di distribuzione allo scaffale del negozio di destinazione}
}
\newglossaryentry{Simulazione di prelievo} {
    name=\glslink{Simulazione di prelievo}{simulazione di prelievo},
    text=simulazione di prelievo,
    sort=simulazione di prelievo,
    description={ una \emph{simulazione di prelievo} è l'esecuzione di una o più missioni di prelievo.
    Il numero di missioni di prelievo da eseguire dipende dal numero di ordini cliente da preparare durante la simulazione
    e dalla dimensione del carrello utilizzato per prelevare gli articoli dal magazzino sotto test in quanto il 
    numero di missioni di prelievo da eseguire è inversamente proporzionale alla dimensione del carrello.
    Nel contesto applicativo del progetto il carrello può contenere un solo ordine per sezione e non vengono tenuti in 
    considerazione altri vincoli dovuti alle proprietà fisiche degli articoli da prelevare.
    Lo scopo principale di una simulazione di prelievo è quindi tenere traccia dello stato delle sezioni del carrello 
    ad ogni fermata per ogni missione di prelievo e fornire all'utente i dati di interesse per valutare la bontà di una
    configurazione di magazzino in termini di numero di fermate effettuate e distanza percorsa al termine di tutte le missioni 
    di prelievo. 
    }}
\newglossaryentry{Missione di prelievo} {
    name=\glslink{Missione di prelievo}{missione di prelievo},
    text=missione di prelievo,
    sort=missione di prelievo,
    description={ una \emph{missione di prelievo} consiste nel completare un numero definito di ordini cliente
    utilizzando un carrello che partendo dall'inizio del magazzino sotto test e percorrendolo
    senza saltare ubicazioni e senza la possibilità di retrocedere verso ubicazioni già
    visitate preleva gli articoli che compongono gli ordini.
    Il numero di ordini cliente coinvolti non può superare la dimensione del carrello utilizzato.
    Nel contesto applicativo del progetto la dimensione del carrello corrisponde al numero massimo di ordini che può 
    ospitare durante una missione di prelievo.
    }}
\newglossaryentry{Articolo associato} {
    name=\glslink{Articolo associato}{articolo associato},
    text=articolo associato,
    sort=articolo associato,
    description={ un \emph{articolo associato} ad un altro articolo si definisce tale quando è presente in almeno un 
    ordine cliente con quell'articolo.
    }
}
